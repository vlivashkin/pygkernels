\documentclass{article}
\usepackage[T2A]{fontenc}
\usepackage[utf8]{inputenc}
\usepackage[english, russian]{babel}

\usepackage{geometry}
\geometry{a4paper, top=1cm, bottom=2cm, left=1.5cm, right=1.5cm}

\usepackage{indentfirst}
\usepackage{amssymb,amsfonts,amsmath}
\usepackage{cite,enumerate}
\usepackage[pdftex,colorlinks,unicode,bookmarks]{hyperref}

\usepackage{floatrow}
\usepackage{caption}
\usepackage{subcaption}


\title{
	Сравнение результатов вычислений моего кода и известных мне статей (анализ тестов)
}
\author{Владимир Ивашкин}

\begin{document}

\maketitle

\section{Введение}
Захотелось написать большой развернутый отчет о всех тестах, которые я провожу. Возможно, это поможет найти наши ошибки и в будущем быть уверенными в результатах.

\section{Chebotarev: Studying new classes of graph metrics}
Ссылка: \url{https://arxiv.org/abs/1305.7514}

Здесь нам интересен Fig. 1. На графе "цепочка" можно прогнать такие же метрики при тех же параметрах. В обозначениях ниже я имею в виду, что вершины графа названы слева направо цифрами от 0 до 3. Важно:
\begin{itemize}
  \item это расстояния, не близости
  \item расстояния здесь нормированы на то, чтобы сумма $D[0, 1] + D[1, 2] + D[2, 3] = 3$
\end{itemize}
Достаточно будет сравнивать расстояния $D[0, 1], D[1, 2], D[0, 2], D[0, 3]$.
Будем считать, что расстояния не соответствуют друг другу, если хотя бы одна соответствующая пара расстояний различается на 0.04 в абсолютной величине.
True --- результат из статьи, Test --- посчитанные результаты, Diff --- абсолютная разница
Итак, вот результаты:

\begin{table}[H]
  \caption{Результаты воспроизведения результатов из Fig. 1}
  \begin{center}
    \begin{tabular}{rc|cccc|c}
                         &      & $D[0, 1]$ & $D[1, 2]$ & $D[0, 2]$ & $D[0, 3]$ & Passed \\
                         \hline
Shortest path,     & True & 1,000       & 1,000       & 2,000       & 3,000       &        \\
Resistance distance  & Test & 1,000       & 1,000       & 2,000       & 3,000       &        \\
                         & Diff & 0,000       & 0,000       & 0,000       & 0,000       & TRUE   \\
                         \hline
Walk, $\alpha=1$ & True & 1,025       & 0,950       & 1,975       & 3,000       &        \\
                         & Test & 1,013       & 0,975       & 1,406       & 1,732       &        \\
                         & Diff & 0,012       & 0,025       & 0,569       & 1,268       & FALSE  \\
                         \hline
log Forest, $\alpha=2$ & True & 0,959       & 1,081       & 2,040       & 3,000       &        \\
                         & Test & 0,980       & 1,040       & 1,429       & 1,733       &        \\
                         & Diff & 0,021       & 0,041       & 0,611       & 1,267       & FALSE  \\
                         \hline
Forest, $\alpha=1$ & True & 1,026       & 0,947       & 1,500       & 1,895       &        \\
                         & Test & 1,013       & 0,973       & 1,225       & 1,377       &        \\
                         & Diff & 0,013       & 0,026       & 0,275       & 0,518       & FALSE  \\
                         \hline
Square-rooted resistance & True & 1,000       & 1,000       & 1,414       & 1,732       &        \\
                         & Test & 1,000       & 1,000       & 1,414       & 1,732       &        \\
                         & Diff & 0,000       & 0,000       & 0,000       & 0,000       & TRUE   \\
                         \hline
Communicability & True & 0,964       & 1,072       & 1,492       & 1,564       &        \\
(не указан параметр!) & Test & 0,964       & 1,072       & 1,492       & 1,564       &        \\
                         & Diff & 0,000       & 0,000       & 0,000       & 0,000       & TRUE   \\
                         \hline
plain Walk $\alpha=4.5$    & True & 1,025       & 0,950       & 1,541       & 1,466       &        \\
                         & Test & 1,013       & 0,975       & 1,242       & 1,211       &        \\
                         & Diff & 0,012       & 0,025       & 0,299       & 0,255       & FALSE  \\
                         \hline
plain Walk, $\alpha=1$       & True & 0,988       & 1,025       & 1,379       & 1,416       &        \\
                         & Test & 0,994       & 1,012       & 1,174       & 1,190       &        \\
                         & Diff & 0,006       & 0,013       & 0,205       & 0,226       & FALSE  \\  
    \end{tabular}
  \end{center}
  \label{cha1:fig1natural}
\end{table}

Итоги более коротко: результаты совпали для Shortest path, Resistance, Square-rooted resistance, Communicability, и не совпали для Walk, plain Walk, Forest, log Forest.

Давайте рассмотрим процесс получения матрицы расстояний на примере log Forest:
\begin{enumerate}
  \item Считаем $H0 = (I + tL)^{-1}$. $t$ --- параметр метрики, в данном случае передается напрямую.
  \item Поэлементно логарифмируем $H = \text{element-wise} \log(H0)$
  \item Превращаем ядро в матрицу расстояний с помощью $D = (h * 1^T + 1 * h^T - H - H ^ T) / 2$, $h$ --- диагональ $H$.
\end{enumerate}

Исследуя разные гипотезы и перепроверяя все что только можно, я наткнулся на следующий интересный факт: ответы начинают совпадать, если конечную матрицу расстояний возвести в квадрат. Причем не для всех метрик, а только для тех, результаты для которых не совпадали раньше. Это странно. В чем здесь может быть дело?

С корректировками таблица выглядит так:

\begin{table}[H]
  \caption{Результаты после возведения некоторых расстояний в квадрат}
  \begin{center}
    \begin{tabular}{rc|cccc|c}
                         &      & $D[0, 1]$ & $D[1, 2]$ & $D[0, 2]$ & $D[0, 3]$ & Passed \\
                         \hline
Shortest path,     & True & 1,000       & 1,000       & 2,000       & 3,000       &        \\
Resistance distance  & Test & 1,000       & 1,000       & 2,000       & 3,000       &        \\
                         & Diff & 0,000       & 0,000       & 0,000       & 0,000       & TRUE   \\
                         \hline
Walk, $\alpha=1$ & True & 1,025       & 0,950       & 1,975       & 3,000       &        \\
                         & Test & 1,025       & 0,950       & 1,975       & 3,000       &        \\
                         & Diff & 0,000       & 0,000       & 0,000       & 0,000       & TRUE   \\
                         \hline
log Forest, $\alpha=2$ & True & 0,959       & 1,081       & 2,040       & 3,000       &        \\
                         & Test & 0,959       & 1,081       & 2,041       & 3,000       &        \\
                         & Diff & 0,000       & 0,000       & 0,001       & 0,000       & TRUE  \\
                         \hline
Forest, $\alpha=1$ & True & 1,026       & 0,947       & 1,500       & 1,895       &        \\
                         & Test & 1,026       & 0,947       & 1,500       & 1,895       &        \\
                         & Diff & 0,000       & 0,000       & 0,000       & 0,000       & TRUE  \\
                         \hline
Square-rooted resistance & True & 1,000       & 1,000       & 1,414       & 1,732       &        \\
                         & Test & 1,000       & 1,000       & 1,414       & 1,732       &        \\
                         & Diff & 0,000       & 0,000       & 0,000       & 0,000       & TRUE   \\
                         \hline
Communicability & True & 0,964       & 1,072       & 1,492       & 1,564       &        \\
(не указан параметр!) & Test & 0,964       & 1,072       & 1,492       & 1,564       &        \\
                         & Diff & 0,000       & 0,000       & 0,000       & 0,000       & TRUE   \\
                         \hline
plain Walk, $\alpha=4.5$    & True & 1,025       & 0,950       & 1,541       & 1,466       &        \\
                         & Test & 1,025       & 0,950       & 1,541       & 1,466       &        \\
                         & Diff & 0,000       & 0,000       & 0,000       & 0,000       & TRUE  \\
                         \hline
plain Walk, $\alpha=1$       & True & 0,988       & 1,025       & 1,379       & 1,416       &        \\
                         & Test & 0,988       & 1,025       & 1,379       & 1,416       &        \\
                         & Diff & 0,000       & 0,000       & 0,000       & 0,000       & TRUE   \\ 
    \end{tabular}
  \end{center}
  \label{cha1:fig1cheated}
\end{table}

Что, помимо вопроса насчет квадрата расстояния, можно вынести из этой истории?
Все меры из этой статьи реализованы верно, функция преобразования параметра $\alpha \rightarrow t$ реализована верно. Также верно реализованы функции-переходы $H0 \rightarrow H$ и $H \rightarrow D$.

\section{Chebotarev: The Walk Distances in Graphs}
Ссылка: \url{https://arxiv.org/abs/1103.2059}

Думая о том, почему не совпадают результаты в предыдущей статье, я нашел также и вот эту статью. В ней есть Table 1, в которой те же метрики при тех же параметрах, но используются отношения расстояний. Воспроизвел также и эти результаты. Проблема здесь точно такая же: часть мер нужно возводить в квадрат, чтобы получился такой же ответ. Результаты в таблице \ref{che2:table1cheated}.

\begin{table}[H]
  \caption{Результаты после возведения некоторых расстояний в квадрат}
  \begin{center}
    \begin{tabular}{rc|ccc|c}
                     &      & $\frac{D[0, 1]}{D[1, 2]}$ & $\frac{D[0, 1] + D[1, 2]}{D[0, 2]}$ & $\frac{D[0, 3]}{D[0, 2]}$ & Passed \\
                     \hline
Shortest path,        & True & 1.000                   & 1.000                                     & 1.500                   &        \\
Resistance distance & Test & 1.000                   & 1.000                                     & 1.500                   &        \\
                     & Diff & 0.000                   & 0.000                                     & 0.000                   & TRUE   \\
                     \hline
Walk, $\alpha=1$ & True & 1.080                   & 1.000                                     & 1.520                   &        \\
                     & Test & 1.080                   & 1.000                                     & 1.519                   &        \\
                     & Diff & 0.000                   & 0.000                                     & 0.001                   & TRUE   \\
                     \hline
Log forest, $\alpha=2$ & True & 0.890                   & 1.000                                     & 1.470                   &        \\
                     & Test & 0.887                   & 1.000                                     & 1.470                   &        \\
                     & Diff & 0.003                   & 0.000                                     & 0.000                   & TRUE   \\
                     \hline
Forest, $\alpha=1$ & True & 1.080                   & 1.320                                     & 1.260                   &        \\
                     & Test & 1.083                   & 1.316                                     & 1.263                   &        \\
                     & Diff & 0.003                   & 0.004                                     & 0.003                   & TRUE   \\
                     \hline
Plain walk, $\alpha=4.5$ & True & 1.080                   & 1.280                                     & 0.950                   &        \\
                     & Test & 1.079                   & 1.281                                     & 0.951                   &        \\
                     & Diff & 0.001                   & 0.001                                     & 0.001                   & TRUE   \\
                     \hline
Plain walk, $\alpha=1$ & True & 0.960                   & 1.460                                     & 1.030                   &        \\
                     & Test & 0.964                   & 1.459                                     & 1.027                   &        \\
                     & Diff & 0.004                   & 0.001                                     & 0.003                   & TRUE  
    \end{tabular}
  \end{center}
  \label{che2:table1cheated}
\end{table}

Скорее всего, это одни и те же данные, но я подумал, что если здесь есть ошибка, то она могла быть исправлена для одной стати и не исправлена для другой.


\section{Kivimaki: Developments in the theory of randomized shortest paths with a article comparison of graph node distances}
Ссылка: \url{https://arxiv.org/abs/1212.1666}

Здесь мы можем использовать два источника: это Figure 2, а также Table 2 с оптимальными значениями из Table 1.

\subsection{Figure 2}
Здесь исследуется поведение метрик RSP, FE, pRes, logFor, SP-CT при изменении их параметров в заданном интервале для графа "треугольник с хвостом". Можно исследовать только крайние точки: слева отношение "\delta_{12}/\delta_{23}" равно 1.5, справа --- 1.0.



\end{document}







