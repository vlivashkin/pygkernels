\documentclass{article}
\usepackage[T2A]{fontenc}
\usepackage[utf8]{inputenc}
\usepackage[english, russian]{babel}

\usepackage{geometry}
\geometry{a4paper, top=1cm, bottom=2cm, left=1cm, right=1cm}

\usepackage{indentfirst}
\usepackage{amssymb,amsfonts,amsmath}
\usepackage{cite,enumerate}
\usepackage[pdftex,colorlinks,unicode,bookmarks]{hyperref}

\usepackage{floatrow}
\usepackage{caption}
\usepackage{subcaption}
\usepackage{multirow}
\usepackage[table]{xcolor}
\usepackage{graphicx}
\usepackage{booktabs}
\usepackage{bm}

\title{
        Воспроизведение результатов статьи в \href{https://github.com/illusionww/pygraphs}{pygraphs}.
}
\author{Владимир Ивашкин}

\begin{document}

\maketitle

\setcounter{section}{1}

\section{Logarithmic vs. plain measures}

Не ясно, в оригинале был RI или ARI. Если был ARI, то он на тот момент был неправильным. Привожу тут оба варианта

\begin{figure}[H]
	\includegraphics[width=.99\linewidth]{pictures/2_1_RI.png}
	\includegraphics[width=.99\linewidth]{pictures/2_1_ARI.png}
	\caption{\label{f_2_1} $G(100, (2)0.2, 0.05)$, RI and ARI respectively}
\end{figure}

\begin{figure}[H]
	\includegraphics[width=.99\linewidth]{pictures/2_2_RI.png}
	\includegraphics[width=.99\linewidth]{pictures/2_2_ARI.png}
	\caption{\label{f_2_2} $G(100, (3)0.3, 0.1)$, RI and ARI respectively}
\end{figure}


\begin{figure}[H]
	\includegraphics[width=.99\linewidth]{pictures/2_3_RI.png}
	\includegraphics[width=.99\linewidth]{pictures/2_3_ARI.png}
	\caption{\label{f_2_3} $G(200, (2)0.3, 0.1)$, RI and ARI respectively}
\end{figure}

\nopagebreak


\newpage
\section{Competition by Copeland's score}
\begin{table}[H]{\small
	\centering
	\begin{tabular}{lrrrrrrrrr}
		\toprule
		\multicolumn{1}{r}{\textbf{Nodes}}         & 100&    100& 100&  100& 200&  200& 200&  200&{\textbf{Sum}} \\
		\multicolumn{1}{r}{\textbf{Classes}}	   &   2&      2&   4&    4&   2&    2&   4&    4&{\textbf{  of}} \\
		\multicolumn{1}{r}{$\bm{p_{\mathbf{out}}}$}& 0.1&	0.15& 0.1& 0.15& 0.1& 0.15& 0.1& 0.15&{\textbf{scores}} \\
		\midrule
logComm  & 10 & 512 & 406 & -122 & 580 & 333 & 152 & 600 & $\bm{2471}$\\
Comm & 4 & 185 & 86 & 448 & 244 & 297 & 442 & 246 & $\bm{1952}$\\
SCCT & 10 & 287 & 188 & 148 & 289 & 238 & 76 & 458 & $\bm{1694}$\\
Heat & 10 & -310 & 86 & 448 & 136 & 332 & 442 & -260 & $\bm{884}$\\
pWalk & -3 & -41 & 86 & 448 & -41 & -106 & 442 & -138 & $\bm{647}$\\
logHeat & 4 & 67 & -16 & -294 & 202 & 332 & -292 & 166 & $\bm{169}$\\
SCT & -6 & 51 & -106 & 148 & -39 & 69 & 76 & -42 & $\bm{151}$\\
logFor & -8 & 33 & -70 & -298 & 3 & -83 & -262 & 50 & $\bm{-635}$\\
FE & 0 & -12 & -104 & -294 & -97 & -102 & -294 & -4 & $\bm{-907}$\\
For & -10 & -560 & 86 & 448 & -568 & -546 & 442 & -260 & $\bm{-968}$\\
RSP & -3 & 92 & -132 & -358 & -107 & -1 & -336 & -124 & $\bm{-969}$\\
Walk & 4 & 20 & -40 & -316 & -144 & -221 & -346 & -98 & $\bm{-1141}$\\
SP-CT & -12 & -324 & -470 & -406 & -458 & -542 & -542 & -594 & $\bm{-3348}$\\
		\bottomrule
	\end{tabular}
	\caption{\label{t_CopComp} Optimal parameters}
}\end{table}

\begin{table}[H]{\small
	\centering
	\begin{tabular}{lrrrrrrrrr}
		\toprule
		\multicolumn{1}{r}{\textbf{Nodes}}         & 100&    100& 100&  100& 200&  200& 200&  200&{\textbf{Sum}}\\
		\multicolumn{1}{r}{\textbf{Classes}}	   &   2&      2&   4&    4&   2&    2&   4&    4&{\textbf{  of}}\\
		\multicolumn{1}{r}{$\bm{p_{\mathbf{out}}}$}& 0.1&	0.15& 0.1& 0.15& 0.1& 0.15& 0.1& 0.15&{\textbf{scores}}\\
		\midrule
logComm & 440 & 501 & 466 & 340 & 398 & 565 & 574 & 582 & $\bm{3866}$\\
SCCT & 263 & 295 & 360 & 184 & 295 & 397 & 438 & 370 & $\bm{2602}$\\
Comm & 109 & 149 & 106 & 120 & 198 & 60 & 168 & 158 & $\bm{1068}$\\
logHeat & 236 & 59 & 80 & 32 & 391 & 11 & 148 & 98 & $\bm{1055}$\\
logFor & -23 & 57 & 148 & 116 & -126 & 44 & 134 & 94 & $\bm{444}$\\
FE & -74 & 80 & 50 & 120 & -30 & 30 & 38 & 52 & $\bm{266}$\\
Walk & -79 & 119 & 114 & 102 & -84 & -4 & 20 & 76 & $\bm{264}$\\
SCT & -27 & 27 & 4 & -32 & 52 & -6 & 36 & 30 & $\bm{84}$\\
pWalk & 45 & 1 & 20 & 10 & -62 & -31 & -10 & 26 &$\bm{-1}$\\
Heat & 296 & -322 & -492 & -445 & 386 & 249 & -215 & -472 & $\bm{-1015}$\\
RSP & -313 & -117 & -16 & 14 & -338 & -268 & -280 & -84 & $\bm{-1402}$\\
SP-CT & -482 & -287 & -250 & 0 & -585 & -460 & -452 & -352 & $\bm{-2868}$\\
For & -391 & -562 & -590 & -561 & -495 & -587 & -599 & -578 & $\bm{-4363}$\\
		\bottomrule
	\end{tabular}
	\caption{\label{t_CopComp} 90th percentiles}
}\end{table}

\newpage
\section{Reject curves}
\begin{table}[H]
	\begin{tabular}{lrrrr}
		\toprule
		Measure & $G(100, (2)0.3, 0.05)$ & $G(100, (2)0.3, 0.1)$ & $G(100, (2)0.3, 0.15)$\\
        (kernel)& Opt. parameter, ARI    & Opt. parameter, ARI   & Opt. parameter, ARI\\
		\midrule
		pWalk	& 0.93,\;\;	1.00	& 0.87,\;\;	0.91	& 0.73,\;\;	0.66\\
		Walk	& 0.93,\;\;	1.00	& 0.67,\;\;	0.91	& 0.70,\;\;	0.65\\
		For		& 0.60,\;\;	0.99	& 0.97,\;\;	0.51	& 0.40,\;\;	0.01\\
		logFor	& 0.70,\;\;	1.00	& 0.40,\;\;	0.93	& 0.10,\;\;	0.68\\
		Comm	& 0.33,\;\;	1.00	& 0.33,\;\;	0.98	& 0.30,\;\;	0.77\\
		logComm	& 0.33,\;\;	1.00	& 0.47,\;\;	$\bm{1.00}$	& 0.57,\;\;	$\bm{0.91}$\\
		Heat	& 0.37,\;\;	1.00	& 0.60,\;\;	0.87	& 0.73,\;\;	0.15\\
		logHeat	& 0.37,\;\;	1.00	& 0.53,\;\;	0.99	& 0.37,\;\;	0.80\\
		SCT		& 0.40,\;\;	1.00	& 0.57,\;\;	0.94	& 0.43,\;\;	0.72\\
		SCCT	& 0.03,\;\;	1.00	& 0.57,\;\;	0.98	& 0.63,\;\;	0.80\\
		RSP		& 0.97,\;\;	1.00	& 0.97,\;\;	0.93	& 0.97,\;\;	0.67\\
		FE		& 0.90,\;\;	1.00	& 0.90,\;\;	0.91	& 0.87,\;\;	0.68\\
		SP-CT	& 0.00,\;\;	0.99	& 0.03,\;\;	0.78	& 0.07,\;\;	0.49\\
		\bottomrule
	\end{tabular}\caption{\label{t_Optt}Optimal family parameters and the corresponding ARI's}
\end{table}

Ошибка была в том, что подобранные параметры из таблицы выше принадлежат к диапазону $[0, 1]$, а значит их нужно преобразовывать к диапазону, специфичному для конкретной метрики. Я же этого не делал.
Вторая ошибка состояла в том, что я использовал тут близости вместо расстояний. Еще тогда, когда я строил их в прошлый раз, я заметил, что по близостям logComm совсем не обгоняет остальные меры, но по расстояниям эффект выраженный. Тут его тоже видно:

\begin{figure}[H] %tb
	\begin{minipage}{.24\textwidth} %49
		\leftfigure{\includegraphics[width=\linewidth]{pictures/4_pWalk.png}}
		\\\centerline{(a) pWalk}
	\end{minipage}
	\begin{minipage}{.24\textwidth} %49
		\leftfigure{\includegraphics[width=\linewidth]{pictures/4_Walk.png}}
		\\\centerline{(b) Walk}
	\end{minipage}
	\begin{minipage}{.24\textwidth} %49
		\leftfigure{\includegraphics[width=\linewidth]{pictures/4_For.png}}
		\\\centerline{(c) For}
	\end{minipage}
	\begin{minipage}{.24\textwidth} %49
		\leftfigure{\includegraphics[width=\linewidth]{pictures/4_logFor.png}}
		\\\centerline{(d) logFor}
	\end{minipage}
    \\[6pt]
	\begin{minipage}{.24\textwidth} %49
		\leftfigure{\includegraphics[width=\linewidth]{pictures/4_Comm_alt.png}}
		\\\centerline{(e) Comm}
	\end{minipage}
	\begin{minipage}{.24\textwidth} %49
		\leftfigure{\includegraphics[width=\linewidth]{pictures/4_logComm_alt.png}}
		\\\centerline{(f) logComm}
	\end{minipage}
	\begin{minipage}{.24\textwidth} %49
		\leftfigure{\includegraphics[width=\linewidth]{pictures/4_Heat.png}}
		\\\centerline{(g) Heat}
	\end{minipage}
	\begin{minipage}{.24\textwidth} %49
		\leftfigure{\includegraphics[width=\linewidth]{pictures/4_logHeat.png}}
		\\\centerline{(h) logHeat}
	\end{minipage}
    \\[6pt]

	\begin{minipage}{.195\textwidth} %49
		\leftfigure{\includegraphics[width=\linewidth]{pictures/4_SCT.png}}
		\\\centerline{(i) SCT}
	\end{minipage}
	\begin{minipage}{.195\textwidth} %49
		\leftfigure{\includegraphics[width=\linewidth]{pictures/4_SCCT.png}}
		\\\centerline{(j) SCCT}
	\end{minipage}
	\begin{minipage}{.195\textwidth} %49
		\leftfigure{\includegraphics[width=\linewidth]{pictures/4_RSP.png}}
		\\\centerline{(k) RSP}
	\end{minipage}
	\begin{minipage}{.195\textwidth} %49
		\leftfigure{\includegraphics[width=\linewidth]{pictures/4_FE.png}}
		\\\centerline{(l) FE}
	\end{minipage}
	\begin{minipage}{.195\textwidth} %49
		\leftfigure{\includegraphics[width=\linewidth]{pictures/4_SPCT.png}}
		\\\centerline{(m) SP-CT}
	\end{minipage}

    \caption{\label{f_Reject}Reject curves for the graph measures under study}
\end{figure}

\begin{figure}[H] %tb
	\begin{minipage}{.56\textwidth}
		\leftfigure{\includegraphics[width=.75\linewidth]{pictures/4_all_alt.png}}
		\\\centerline{(a) All families}
	\end{minipage}%
	\begin{minipage}{.56\textwidth}
		\leftfigure{\includegraphics[width=.75\linewidth]{pictures/4_4best_alt.png}}
		\\\centerline{(a) All families}
	\end{minipage}%
\caption{\label{f_Rcur}Average reject curves}
\end{figure}

Здесь была проблема со взятием корня из Comm/logComm. А проблема была такая: если некоторые значения матрицы D при взятии корня превращаются в nan, то стандартная сортировка оставляет их на тех же позициях и отдельно сортирует массив слева и справа от них. Получается кусочно-возрастающая функция, из которой потом получаются несколько маленьких reject curve вместо одной большой. Решение -- фильтровать эти nan и сортировать без них. Раз такой эффект вообще возник, значит в матрице D иногда появляются отрицательные значения.

Можно подозревать внешний вид графика pWalk. Может быть, это связано с тем, как мы фиксируем параметр. Параметром считаем отскалированое в [0, 1] число, для каждого графа преобразуем его в зависимости от спектрального радиуса матрицы $A$ ($ param = t / \rho(A)), t \in [0, 1]$).


\newpage
\section{Graphs with classes of different sizes}
\begin{figure}[H]
	\begin{minipage}{.5\textwidth}
		\leftfigure{\includegraphics[width=.9\linewidth]{pictures/5_best1.png}} %0.8
		\\\centerline{(a) All families}
	\end{minipage}%
	\begin{minipage}{.5\textwidth}
		\leftfigure{\includegraphics[width=.9\linewidth]{pictures/5_best2.png}} %1.07
		\\\centerline{(b) Leading families}
	\end{minipage}
\caption{\label{f_difClas}Graphs with two classes of different sizes: clustering with optimal parameter values}
\end{figure}

\begin{figure}[H]
	\leftfigure{\includegraphics[width=.45\linewidth]{pictures/5_avg.png}}
\caption{\label{f_difClas1}Graphs with two classes of different sizes: random parameter values}
\end{figure}

\begin{figure}[H]
\samenumber
\begin{minipage}{.45\textwidth}
{\normalsize
$$
P=\begin{pmatrix}
    0.30& 0.20& 0.10& 0.15& 0.07& 0.25\\
    0.20& 0.24& 0.08& 0.13& 0.05& 0.17\\
    0.10& 0.08& 0.16& 0.09& 0.04& 0.12\\
    0.15& 0.13& 0.09& 0.20& 0.02& 0.14\\
    0.07& 0.05& 0.04& 0.02& 0.12& 0.04\\
    0.25& 0.17& 0.12& 0.14& 0.04& 0.40\\
  \end{pmatrix}.
$$}
\end{minipage}
\begin{minipage}{.45\textwidth}
	\leftfigure{\includegraphics[width=.85\linewidth]{{pictures/5_six}.png}}
\end{minipage}
\twocaptionwidth{.45\textwidth}{.45\textwidth}\phantom{\rightcaption{}}\rightcaption{\label{f_6classes}ARI of various measure families on a structure with 6 classes}
\end{figure}


\newpage
\section{Cluster analysis on several classical datasets}

Здесь ошибка была в том, что я зафиксировал число классов -- 2, хотя в датасете football их 12.
Теперь все похоже на статью:

\begin{figure}[H]
	\begin{minipage}{.32\textwidth}
		\leftfigure{\includegraphics[width=\linewidth]{pictures/6_football.png}}
		\\\centerline{(a) football}
	\end{minipage}
	\begin{minipage}{.32\textwidth}
		\leftfigure{\includegraphics[width=\linewidth]{pictures/6_polbooks.png}}
		\\\centerline{(b) polbooks}
	\end{minipage}
	\begin{minipage}{.32\textwidth}
		\leftfigure{\includegraphics[width=\linewidth]{pictures/6_zachary.png}}
		\\\centerline{(c) Zachary}
	\end{minipage}
    \\[10pt]
	\begin{minipage}{.32\textwidth}
		\leftfigure{\includegraphics[width=\linewidth]{pictures/6_news_2cl_1.png}}
		\\\centerline{(d) news\_2cl\_1}
	\end{minipage}
	\begin{minipage}{.32\textwidth}
		\leftfigure{\includegraphics[width=\linewidth]{pictures/6_news_2cl_2.png}}
		\\\centerline{(e) news\_2cl\_2}
	\end{minipage}
	\begin{minipage}{.32\textwidth}
		\leftfigure{\includegraphics[width=\linewidth]{pictures/6_news_2cl_3.png}}
		\\\centerline{(f) news\_2cl\_3}
	\end{minipage}
	\\[10pt]
    \begin{minipage}{\textwidth}
        \hfill\includegraphics[width=0.7\linewidth]{pictures/6_legend.png}\hfill
	\end{minipage}
  \caption{\label{f_datasets}ARI of various measure families on classical datasets}
\end{figure}

\end{document}
